% Package for translation command (see below)
\usepackage{iflang}

% Translation command for French/German/Italian
\newcommand{\trans}[3]{\IfLanguageName{french}{#1}{\IfLanguageName{nswissgerman}{#2}{#3}}}

% Package for total pagenumbers (used in footer)
\usepackage{lastpage}

% No figure numbering
\usepackage[labelformat=empty]{caption}

% Prevent orphans/widows
\usepackage[all]{nowidow}

% Tables and tabulars
\usepackage{tabto}
\usepackage{longtable,ltxtable}
\usepackage{tabu,array,multirow,multicol,adjustbox,booktabs}
\usepackage{colortbl}
\usepackage{makecell}
\usepackage{tabularx}
\newcolumntype{C}{>{\centering\arraybackslash}X}
\newcolumntype{R}{>{\raggedleft\arraybackslash}X}
\usepackage{hhline}

% Package for math
\usepackage{fp}

% Package for special itemize
\usepackage{enumitem}

% Package for the blacksquare symbol
\usepackage{amsmath}
\usepackage{amssymb}
\usepackage{amsthm}

% Avoid large space to equalise content
\raggedbottom
\flushbottom

% Package for \uline{}
\usepackage[normalem]{ulem}
\renewcommand\ULthickness{2pt}%---> For changing thickness of underline
\setlength\ULdepth{1.5ex}%\maxdimen ---> For changing depth of underline
\renewcommand{\baselinestretch}{1}
\pagestyle{empty}

% Setup examdoc points
\pointsinleftmargin
\pointpoints{\trans{pt}{Pt}{pto}}{\trans{pts}{Pte}{pti}}
\marginpointname{ \points}
\pointformat{\footnotesize\boldmath\themarginpoints}
%\boxedpoints
\usehorizontalhalf
\addpoints

% Remove examdoc solution header
\renewcommand{\solutiontitle}{}

% Provide a box for
\renewcommand{\tftext}[1]{%
\tabto{\CurrentLineWidth}%
\parbox{minipage}[t][][t]{\linewidth-\TabPrevPos}{#1}}

% Setup solutions with no breaks
\BeforeBeginEnvironment{solutionorlines}{\par\nopagebreak\minipage{\linewidth}}
\AfterEndEnvironment{solutionorlines}{\endminipage}
\BeforeBeginEnvironment{solutionorbox}{\par\nopagebreak\minipage{\linewidth}}
\AfterEndEnvironment{solutionorbox}{\endminipage}

% Command for section point table
\newcommand{\sectionpts}[1]{%
	\ifprintanswers
	\else
		\bgroup
		\def\arraystretch{2}
		\begin{tabularx}{.5\textwidth}{|>{\bfseries}X|>{\bfseries}C|}
			\hline
			\trans{Points maximum}{Mögliche Punkte}{Punteggio possibile} & \pointsinrange{#1} \\
			\hline
			\trans{Points obtenus}{Erreichte Punkte}{Punteggio raggiunto} &\\
			\hline
		\end{tabularx}
		\egroup
	\fi
}

% Command for fillin
\newcommand{\lng}[2][1cm]{%
  \fillin[#2][#1]%
}

% Command for TRUE/FALSE
\newcommand{\tf}[1]{%
  \fillin[#1][0.8cm]\tabto{1cm}%
}

% Total point command
\makeatletter
\newcommand{\calcnumpoints}{\@ifundefined{exam@numpoints}{0}{\exam@numpoints}}
\makeatother

% Setup page geometry
\usepackage[bindingoffset=0in,left=1.8cm,right=1.8cm,
top=1cm,bottom=1in,headsep=.5\baselineskip]{geometry}